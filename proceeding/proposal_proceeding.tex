% Please make sure you insert your
% data according to the instructions in PoSauthmanual.pdf
\documentclass[a4paper,11pt]{article}
\usepackage{pos}

\usepackage{xfrac}
\usepackage{siunitx}
\usepackage[size=tiny]{todonotes}
\presetkeys{todonotes}{color=blue!30}{}

\usepackage{amsmath}

% this is for making a more compact reference section
\usepackage{bibspacing}
\setlength{\bibitemsep}{.2\baselineskip plus .05\baselineskip minus .05\baselineskip}


\ShortTitle{Physics updates of the simulation tool PROPOSAL}
\title{Physics updates of the high-energy lepton and photon simulation tool PROPOSAL}

\author*[a]{Jean-Marco Alameddine}
\author[a]{Pascal Gutjahr}
\author[a]{Wolfgang Rhode}
\author[b]{Alexander Sandrock}
\author[a]{Jan Soedingrekso}

\affiliation[a]{Technische Universität Dortmund, Fakultät Physik,\\
  Otto-Hahn-Straße 4a, 44227 Dortmund, Germany}

\affiliation[b]{Bergische Universität Wuppertal,
  Fakultät für Mathematik und Naturwissenschaften,\\
  Gaußstraße 20, 42119 Wuppertal, Germany}


% Uncomment \onbehalf{...} for collaboration if you want.
%\onbehalf{for the XXXX collaboration} 

% In this case, you also have to uncomment the lines after "%Full authors list" below and include the full authors list,

\emailAdd{jean-marco.alameddine@tu-dortmund.de}
\emailAdd{pascal.gutjahr@tu-dortmund.de}
\emailAdd{wolfgang.rhode@tu-dortmund.de}
\emailAdd{asandrock@icecube.wisc.edu}
\emailAdd{jan.soedingrekso@tu-dortmund.de}

\abstract{
Monte Carlo simulations are an important tool in modern physics experiments. With improving detector sensitivities, higher accuracies are also required from simulations, for example in reconstruction tasks. This includes both correctness from a physical as well as an algorithmic point of view. PROPOSAL is a Monte Carlo simulation framework that provides three-dimensional simulations of high-energy photons, electrons, muons, and taus. It is written in C++, but can also be used within Python via a wrapper. The structure of the software allows for simple customization of the propagation environment, physics descriptions, or precision settings for a variety of use cases. Examples are the application in neutrino observatories, underground experiments, or air shower simulations. This contribution focuses on the recent physics updates of the framework, describing the methodology and implication of these improvements. In particular, this involves effects at the higher and lower end of the energy scale covered by PROPOSAL. For high-energy photons, photon-nucleon interactions, muon pairproduction, and the Landau-Pomeranchuk-Migdal effect in electron-positron pairproduction are now included. As lower-energy effects, the deflection of muons in stochastic interactions as well as an approximate description of the photoeffect for photons have been implemented.}

\ConferenceLogo{PoS_ICRC2023_logo.pdf}

\FullConference{%
38th International Cosmic Ray Conference (ICRC2023)\\
  26 July - 3 August, 2023\\
  Nagoya, Japan}



%% \tableofcontents

\begin{document}
\maketitle


\section{Introduction}

Modern experiments in physics rely on methods of statistical data analysis to interpret their measurements.
To train these methods, a statistical sufficient dataset where the true properties are known is required.
Especially in astroparticle physics, this can only be achieved using simulations which model the reality.
One of these simulation tools is the framework PROPOSAL, which provides a three-dimensional Monte Carlo simulation of high-energy particles propagating through large volumes. \footnote{PROPOSAL is available as an open-source C\texttt{++}/Python software under \url{https://github.com/tudo-astroparticlephysics/PROPOSAL}. It can be installed with \texttt{pip install proposal} or via CMake.}
Originally, PROPOSAL has been written for the simulation of muon and tau leptons in the context of underground observatories, such as the IceCube Neutrino Observatory \cite{IceCube:2021uhz}, or for radio neutrino detectors \cite{PhysRevD.102.083011}.
In a recent update, the simulation of electrons, positrons, and photons has been included as well, which allows PROPOSAL to be used to simulate the electromagnetic component of extensive air showers, as it is done in the air shower simulation framework CORSIKA~8 \cite{icrc2023}. 
As a first iteration, the most important processes for the simulation of electromagnetic cascades, namely pair production, annihilation, and Compton scattering, as well as dedicated parametrization for ionization and bremsstrahlung losses have been implemented \cite{Alameddine:2021iq}.
In this contribution, the implementation of additional photon interaction processes is presented.
This includes photoelectric absorption, photonuclear interactions, muon pair production, and a description of the Landau-Pomeranchuk-Migdal suppression in electron-positron pair production.
These processes are important to describe photons as low energies ($E \lessapprox \SI{0.1}{\mega\electronvolt}$ in air) and very-high energies ($E \gtrapprox \SI{1e12}{\mega\electronvolt}$ in air), as well as due to their distinct event signatures.
\section{Physics improvements for photon interactions}

\begin{figure}
	\centering
    \includegraphics{plots/Photon_Air_dndx_ecut_0.pdf}
    \caption{Total cross section of photons in air inside PROPOSAL.}
    \label{fig:total_cross_photon}
\end{figure}


\subsection{Photoelectric absorption}

Photoelectric absorption describes the ejection of an atomic electron due to its interaction with an in-going photon.
In this process, the photon energy is used to free the electron from its atomic binding, while the remaining photon energy serves as the kinetic energy of the now free electron.
For photons in air, photoelectric absorption becomes the dominant interaction process for energies below $\approx \SI{30}{\kilo\electronvolt}$.
In the context of electromagnetic cascades, photoelectric absorption as a process starts to become important for energies where its cross section represents a significant correction to the total mean free path length.

The detailed description of photoelectric absorption is non-trivial and dependent on the properties of the atomic structure of the interaction target.
Since other codes to describe these processes already exist, PROPOSAL only provides an approximate description based on the cross section given in \cite{heitler, sauter}.
The total cross section is defined as
%
\begin{align}
	\label{eqn:sauter}
	\sigma &= 4 \pi r_e^2 Z^5 \alpha^4 F_1 F_2 \left( \frac{m_e}{E} \right)^5 \left( \gamma^2 -1 \right)^{\sfrac{3}{2}} \left[ \frac{4}{3} + \frac{\gamma (\gamma - 2)}{\gamma + 1} \left( 1 - \frac{\ln{\left( \gamma + \sqrt{\gamma^2 - 1} \right)}}{\gamma \sqrt{\gamma^2 - 1}}  \right) \right],
\end{align}
%
with the photon energy $E$ and the definitions
\begin{align}
	\gamma &= 1 + \frac{E - I}{m_e}, & I &= \frac{Z^2 \alpha^2 m_e}{2}.
\end{align}
%
The term
\begin{align}
	F_1 &= \left[ 1 + \left( \frac{\alpha Z}{\beta} \right)^2 \right] \frac{\pi \alpha Z / \beta }{\sinh(\pi \alpha Z / \beta )} \exp\left[ \frac{\alpha Z}{\beta} \left( \pi - 4 \arctan\left( \frac{\beta}{\alpha Z} \right) \right) \right]
\end{align}
%
is used as a correction factor for the non-relativistic energy regime \cite{sauter}, while the term
%
\begin{align}
	F_2 &= 1 + 0.01481 \ln^2{Z} - 0.000788 \ln^3{Z}
\end{align}
%
is an empirical correction describing the ratio between the K-shell and total photoelectric absorption cross section \cite{hubbell1969}.

The photoelectric cross section and a validation of the total photon cross section at low energies is shown in Figure \ref{fig:photoeffect_nist}, where the total photon cross section in air is compared to the calculations from the NIST Standard Reference Database.
The agreement is at worst \SI{10}{\percent}.

\begin{figure}
	\centering
    \includegraphics{plots/photoeffect_nist.pdf}
    \caption{Photon cross sections in air for small energies, compared to the total cross section according to the NIST Standard Reference Database.}
    \label{fig:photoeffect_nist}
\end{figure}

\subsection{Photonuclear interactions}


\begin{align}
	\label{eqn:photonuclear_C7}
	\sigma_{\gamma,N} &=
	\begin{cases}
		\left(73.3 s^{0.073} + 191.7 s^{-0.602} \right) \sqrt{1 - s_0 / s} \si{\micro\barn}, & \text{for } \sqrt{s} \leq \SI{19.39}{\giga\electronvolt}, \\
		\left( 59.3 s^{0.093} + 120.2 s^{-0.358} \right) \si{\micro\barn}, & \text{for } \sqrt{s} > \SI{19.39}{\giga\electronvolt}, \\
	\end{cases}
\end{align} 

with the squared center of mass energy $s = m_n^2 + 2 m_n \nu$ and the pion production threshold $\sqrt{s_0} = \SI{1.0761}{\giga\electronvolt}$.

\todo{Shadowing???}

\begin{figure}
	\centering
    \includegraphics{plots/photoproduction_cross.pdf}
    \caption{Photonuclear interaction cross sections.}
    \label{fig:photoproduction_cross}
\end{figure}


\subsection{Muon pair production}

\begin{align}
	\frac{\mathrm{d}\sigma}{\mathrm{d}x} &= 4 Z^2 \alpha \left( r_e \frac{m_e}{m_\mu} \right)^2 \Phi(\delta) \left[ 1 - \frac{4}{3} (x - x^2) \right],
\end{align}
%
with
%
\begin{align}
	\Phi(\delta) &= \underbrace{\ln \left( \frac{B Z^{\sfrac{-1}{3}} m_\mu / m_e}{1 + B Z^{\sfrac{-1}{3}} \sqrt{e} \delta / m_e } \right)}_{\Phi_0} - \underbrace{\ln\left( \frac{D_n}{1 + \delta (D_n \sqrt{e} - 2) / m_\mu} \right)}_{\Delta_\text{n}},
\end{align}
%
where
%
\begin{align}
	x &= \frac{E_{\mu^-}}{E}, & \delta &= \frac{m_\mu^2}{2 E x (1 - x)}, & D_n &= 1.54 A^{0.27}.
\end{align}
%

\begin{align}
	\Phi(\delta) &\rightarrow \Phi(\delta) + \frac{1}{Z} \left[ \ln\left( \frac{m_\mu / \delta}{\delta m_\mu / m_e^2 + \sqrt{e}} \right) - \ln\left( 1 + \frac{1}{\delta \sqrt{e} B^\prime Z^{\sfrac{-2}{3}} / m_e} \right) \right]
\end{align}

For $Z > 1$, the effect of the inelastic nuclear form factor is taken into account by substituting
%
\begin{align}
	\Delta_\text{n} &\rightarrow \left( 1 - \frac{1}{Z} \right) \Delta_\text{n}.
\end{align}

\subsection{Landau-Pomeranchuk-Migdal effect}

\begin{align}
	\label{eqn:lpm_photopair}
	\frac{\mathrm{d}\sigma_\text{LPM}}{\mathrm{d}x} &= \frac{\mathrm{d}\sigma}{\mathrm{d}x} \cdot \frac{\xi(s) / 3 \left(G(s) + 2 \left( x^2 + (1 - x)^2 \right) \phi(s) \right)}{1 - 4 / 3 x (1 - x)},
\end{align}

\begin{align}
	\phi(s) &=
	\begin{cases}
		1 - \exp{\left\{ -6s \left(1 + (3 - \pi) s\right) + \frac{s^3}{ 0.623 + 0.796s + 0.658 s^2} \right\}} & \text{if $s < \num{1.54954}$}, \\
		1 - 0.012 s^{-4} & \text{if $s \geq \num{1.54954}$},
	\end{cases}
\end{align}
%
\begin{align}
	G(s) &=
	\begin{cases}
		3\psi(s) - 2\phi(s) & \text{if $s < \num{0.710390}$}, \\
		36s^2 / \left(36s^2 + 1 \right) & \text{if $\num{0.710390} \leq s < \num{0.904912}$}, \\
		1 - 0.022s^{-4} & \text{if $s \geq \num{0.904912}$},
	\end{cases}
\end{align}
%
\begin{align}
	\psi(s) &= 1 - \exp{\left\{ -4s - \frac{8s^2}{1 + 3.936s + 4.97s^2 - 0.05s^3 + 7.5 s^4} \right\}},
\end{align}
%
\begin{align}
	\xi(s) &\approx \xi(s^\prime) =
	\begin{cases}
		2 & \text{if $s^\prime < s_1$}, \\
		1 + h - \frac{0.08 (1 - h) (1 - (1-h)^2)}{\ln{(s_1)}} & \text{if $s_1 \leq s^\prime < 1$}, \\
		1 & \text{if $s^\prime \geq 1$},
	\end{cases}
\end{align}


\begin{align}
	s &= \frac{s^\prime}{\sqrt{\xi(s^\prime)}}, & s^\prime &= \frac{1}{8} \sqrt{\frac{E_\text{LPM}}{E x ( 1 - x)}}, & s_1 &= \frac{\sqrt{2} Z^{\sfrac{2}{3}}}{B^2}, \\ E_\text{LPM} &= \frac{2 \alpha (m_e c^2)^2 X_0}{\pi \hbar c}, & h &= \frac{\ln{(s^\prime)}}{\ln{(s_1)}}, & D_n &= 1.54 A^{0.27},
\end{align}

\begin{figure}
\centering
\begin{minipage}[t]{.48\textwidth}
  \centering
  \includegraphics{plots/lpm_photopair_differential_small.pdf}
  \captionof{figure}{Differential cross section for electron-positron pair production in air at standard density, with nitrogen as an interaction target. The effect of the LPM effect at different energies is shown. Note that without the LPM suppression, the differential cross section is identical for all energies in this plot.}
  \label{fig:lpm_photopair_diff}
\end{minipage}%
\hfill
\begin{minipage}[t]{.48\textwidth}
  \centering
  \includegraphics{plots/lpm_cross_photopair_small.pdf}
  \captionof{figure}{Total cross section for electron-positron pair production in air, with the effect of the LPM suppression at different atmospheric heights.}
  \label{fig:lpm_photopair_cross}
\end{minipage}
\end{figure}
\section{Summary and outlook}

In this contribution, the implementation of the photoelectric effect, photonuclear interactions, muon pair production, and the LPM effect for pair production as well as their effect on the total cross section of photons have been presented.
For photon energies below $\approx \SI{20}{\mega\electronvolt}$, the implementation has been verified by comparison with independent calculations. 
With these updates, PROPOSAL is now able to provide a complete description of electron, positron, and photon processes in the energy range important for the simulation of electromagnetic cascades.
The current main application for these improvements is the air shower simulation framework CORSIKA~8, where the current status is presented in \cite{icrc2023}.
However, the modular structure allows its usage for many other applications as well.
In the future, the simulation of neutrino interactions with PROPOSAL using only-stochastic propagation is planned to be implemented.

\acknowledgments
This work has been supported by the Deutsche Forschungsgemeinschaft (DFG) and the Lamarr institute.
Jean-Marco Alameddine acknowledges the financial support by the German Academic Exchange Service (DAAD).

\bibliographystyle{JHEP}
\footnotesize
\bibliography{references}


%% Full authors list (ONLY FOR COLLABORATIONS)
%\clearpage
%\section*{Full Authors List: \Coll\ Collaboration}
%
%\noindent \textbf{Note comment afterwards:} Collaborations have the possibility to provide an authors list in xml format which will be used while generating the DOI entries making the full authors list searchable in databases like Inspire HEP. \\
%
%\scriptsize
%\noindent
%first.author$^1$, 
%second.author$^2$, 
%third.author$^3$ % .... more names
%and 
%last.author$^{n}$ \\
%
%\noindent
%$^1$first.affiliation.
%$^2$second.affiliation. % .... more affiliation
%$^{m}$last.affiliation.

\end{document}
